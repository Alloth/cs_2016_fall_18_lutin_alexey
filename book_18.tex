\documentclass{book}
\usepackage[english]{babel} 
\usepackage[utf8x]{inputenc} 
\usepackage[T1]{fontenc} 
\usepackage{fancyhdr} 
\usepackage[a4paper,top=2cm,bottom=2cm,left=3cm,right=2cm,marginparwidth=1cm]{geometry} 
\usepackage{xcolor}  
\usepackage{amsmath} 
\usepackage{graphicx} 
\usepackage[colorinlistoftodos]{todonotes} 
\usepackage[colorlinks=true, allcolors=blue]{hyperref} 
\usepackage{setspace} 
\setcounter{chapter}{4}
\setcounter{section}{4}
\setlength{\headheight}{14pt}
\newcounter{pro1}
\setcounter{pro1}{30}
\newcommand{\pro}{\par\addtocounter{pro1}{1}
\textbf{Problem \arabic{chapter}.\arabic{pro1} }\quad}
\setcounter{equation}{20}
\definecolor{light-gray}{rgb}{0.8,0.8,0.8}

\begin{document}
\pagestyle{fancy} 

 
\renewcommand{\headrulewidth}{0pt}
\fancyhf{} 

\fancyhead[RO]{\large \textsl {\textbf{65}}} 
\fancyhead[LO]{\large \textsl{4.2 Arithmetic and geometric means}} 

\noindent\Large \textrm{Fortunately, a surprising trigonometric identity due to John Machin (1686–1751)}

\begin {equation}\label{formula4.21}
\arctan1=4\arctan\frac{1}{5}-\arctan\frac{1}{239}
 \end{equation}
\\
\Large \textrm{accelerates the convergence by reducing x:}

\begin {equation}\label{formula4.22}
\frac{\pi}{4}=4\times\underbrace{(1-\frac{1}{3\times 5^{3}}+\dots)}_{\arctan(1/5)}-\underbrace{(1-\frac{1}{3\times 239^{3}}+\dots)}_{\arctan(1/239)}\cdot
\end{equation}
\\
\noindent \Large \textrm{ Even with the speedup, $10^{9}$-digit accuracy requires calculating roughly $10^{9}$ terms.}\\
\\
\noindent\Large \textrm{ In contrast, the modern Brent–Salamin algorithm [3, 41], which relies onarithmetic and geometric means, converges to $\pi$ extremely rapidly. The
algorithm is closely related to amazingly accurate methods for calculating
the perimeter of an ellipse (Problem 4.15) and also for calculating mutual
inductance [23]. The algorithm generates several sequences by starting
with $a_{0}$ = 1 and $g_{0}$ = 1/√2; it then computes successive arithmetic means
an, geometric means $g_{n}$, and their squared differences $d_{n}$.}


\begin {equation}\label{formula4.23}
a_{n+1}=\frac{a_{n}+g_{n}}{2}{,}\;\;\;\;g_{n+1}=\sqrt{a_{n}g_{n}}{,}\;\;\;\;d_{n}=a_n^2-g_n^2\cdot
\end{equation}
\\
\noindent\Large \textrm{The a and g sequences rapidly converge to a number M($a_{0}$, $g_{0}$) called the arithmetic–geometric mean of $a_{0}$ and $g_{0}$. Then M($a_{0}$, $g_{0}$) and the difference sequence d determine $\pi$.}
\begin {equation}\label{formula4.24}
\pi=\frac{4M(a_{0},g_{0})^{2}}{1-\sum_{j=1}^\infty 2^{j+1}d_{i}}\cdot
\end{equation}
\\
\noindent\Large \textrm{ The d sequence approaches zero quadratically; in other words, $d_{n+1}\sim d_n^2$ (Problem 4.16). Therefore, each iteration in this computation of $\pi$ doubles the digits of accuracy. A billion-digit calculation of $\pi$ requires only about
30 iterations—far fewer than the $10^{10^{9}}$ terms using the arctangent series
with x=1 or even than the $10^{9}$ terms using Machin’s speedup.} \\
\\
\colorbox{light-gray}{
\begin{minipage}{\textwidth}
\Large \textrm{\textbf{Problem 4.15 \;  Perimeter of an ellipse}\\
To compute the perimeter of an ellipse with semimajor axis $a_{0}$ and semiminor
axis $g_{0}$, compute the a, g, and d sequences and the common limit M($a_{0}$, $g_{0}$) of
the a and g sequences, as for the computation of $\pi$. Then the perimeter P can
be computed with the following formula:}\\
\end{minipage}}
\newpage
\pagestyle{fancy} 

\renewcommand{\headrulewidth}{0pt} 
\fancyhf{}

\fancyhead[LE]{\large \textsl {\textbf{66}}} 
\fancyhead[RE]{\large \textsl{4 Pictorial proofs}}
\colorbox{light-gray}{
\begin{minipage}{\textwidth}
\begin {equation}\label{formula4.25}
P=\frac{A}{M(a_{b0},g_{0})}\left(a_0^2-B\sum_{j=0}^\infty 2^{j}d_{j}\right){,}
\end{equation}
\large\textrm{where A and B are constants for you to determine. Use the method of easy cases
(Chapter 2) to determine their values. (See [3] to check your values and for a
proof of the completed formula.)}\\
\\
\Large \textrm{\textbf{Problem 4.16 \;  Quadratic convergence}\\
Start with $a_{0}$ = 1 and $g_{0}$ = 1/$\sqrt{2}$ (or any other positive pair) and follow severaliterations of the AM–GM sequence}\\
\\
\begin {equation}\label{formula4.26}
a_{n+1}=\left(\frac{a_{n}+g_{n}}{2}\right)\;\;and\;\;a_{n+1}=\sqrt{a_{n}g_{n}}_\cdot
\end{equation}
\\
\large\textrm{Then generate $d_{n}=a_n^2-g_n^2$  and  $\log_{10}{d_{n}}$ to check that $d_{n+1}\sim d_n^2$ (quadratic convergence).}\\
\\
\Large \textrm{\textbf{Problem 4.17 \;  Rapidity of convergence}\\
Pick a positive $x_{0}$; then generate a sequence by the iteration}\\
\\
\begin {equation}\label{formula4.27}
x_{n+1}=\frac{1}{2}\left(x_{n}+\frac{2}{x_{n}}\right)\;\;(n\geq0)\cdot
\end{equation}
\\
\large\textrm{To what and how rapidly does the sequence converge? What if $x_{0}<0$}\\
\end{minipage}}\\
\\
\Large \textrm{\textbf{4.3 \; Approximating the logarithm}\\
A function is often approximated by its Taylor series}\\
\begin {equation}\label{formula4.28}
f(x)=f(0)+\left. \ x\frac{\text{d}f}{\text{d}x}\right|_{x=0}+\frac{x^{2}}{2}\left.\frac{\ d^{2}f}{\text{d}x^{2}}\right|_{x=0}+\dots{,}
\end{equation}
\large\textrm{which looks like an unintuitive sequence of symbols.
Fortunately, pictures often explain the first and most
important terms in a function approximation. For example, the one-term
approximation $\sin\theta\approx\theta$, which replaces the altitude of the triangle by
the arc of the circle, turns the nonlinear pendulum differential equation
into a tractable, linear equation (Section 3.5).}\\
\\
\large\textrm{Another Taylor-series illustration of the value of pictures come from the
series for the logarithm function:}\\
\\
\begin {equation}\label{formula4.29}
\ln{1+x}=x-\frac{x^{2}}{2}+\frac{x^{3}}{3}-\dots\cdot
\end{equation}
\newpage
\fancyhf{} 

\fancyhead[RO]{\large \textsl {\textbf{67}}} 
\fancyhead[LO]{\large \textsl{4.3 Approximating the logarithm}}

\noindent\Large \textrm{Its first term, x, will lead to the wonderful approximation $(1+x)^{n}\approx e^{nx}$ for small x and arbitrary n (Section 5.3.4). Its second term, −$x^{2}$/2, helps
evaluate the accuracy of that approximation. These first two terms are
the most useful terms—and they have pictorial explanations.}\\
\\
\Large \textrm{The starting picture is the integral representation}\\
\begin {equation}\label{formula4.30}
\ln{(1+x)}=\int_{0}^{x}\frac{\text{d}t}{\text{1+t}}\cdot
\end{equation}
\large \textsl{What is the simplest approximation for the shaded area?}\\
\\
\Large \textrm{As a first approximation, the shaded area is roughly
the circumscribed rectangle—an example of lumping.
The rectangle has area x:}\\
\begin {equation}\label{formula4.31}
area=\underbrace{height}_{1}\times\underbrace{width}_{x}=x_\cdot
\end{equation}
\Large \textrm{This area reproduces the first term in the Taylor series. Because it uses a
circumscribed rectangle, it slightly overestimates $\ln{(1+x)}$.}\\
\\
\Large \textrm{The area can also be approximated by drawing an inscribed
rectangle. Its width is again x, but its height
is not 1 but rather 1/(1+x), which is approximately
1 − x (Problem 4.18). Thus the inscribed rectangle
has the approximate area x(1 − x) = x − $x^{2}$. This
area slightly underestimates $\ln{(1+x)}$.}\\
\\
\colorbox{light-gray}{
\begin{minipage}{\textwidth}
\Large \textrm{\textbf{Problem 4.18 \;  Picture for approximating the reciprocal function}\\
Confirm the approximation}
\begin {equation}\label{formula4.32}
\frac{1}{(1+x)}\approx 1-x \;\; (for\;small\;x)
\end{equation}
by trying x = 0.1 or x = 0.2. Then draw a picture to illustrate the equivalent
approximation (1 − x)(1 + x) $\approx $ 1.
\end{minipage}}\\
\\
\Large \textrm{We now have two approximations to $\ln{(1+x)}$. The first and slightly
simpler approximation came from drawing the circumscribed rectangle.
The second approximation came from drawing the inscribed rectangle.
Both dance around the exact value.}\\
\\
\large \textsl{How can the inscribed- and circumscribed-rectangle approximations be combined
to make an improved approximation?}\\
\\
\newpage
\fancyhf{}

\fancyhead[LE]{\large \textsl {\textbf{68}}} 
\fancyhead[RE]{\large \textsl{4 Pictorial proofs}}
\noindent\Large \textrm{One approximation overestimates the area, and the
other underestimates the area; their average ought
to improve on either approximation. The average is
a trapezoid with area}\\
\begin {equation}\label{formula4.33}
\frac{x+(x-x^{2})}{2}=x-\frac{x^{2}}{2}\cdot
\end{equation}
\Large \textrm{This area reproduces the first two terms of the full Taylor series}\\
\begin {equation}\label{formula4.34}
\ln{(1+x)}=\colorbox{light-gray}{$x-\frac{x^{2}}{2}$}+\frac{x^{3}}{3}-\cdots\cdot
\end{equation}
\colorbox{light-gray}{
\begin{minipage}{\textwidth}
\Large \textrm{\textbf{Problem 4.19 \;  Cubic term}\\
Estimate the cubic term in the Taylor series by estimating the difference between
the trapezoid and the true area.}
\end{minipage}}\\
\\
\Large \textrm{For these logarithm approximations, the hardest problem is $\ln{2}$.}\\
\begin {equation}\label{formula4.35}
\ln{(1+1)}=\begin{cases}1\;\;\;\;\;\;\;\;(one\;term)\\1-\frac{1}{2}\;\;(two\;terms)\end{cases}
\end{equation}
\Large \textrm{Both approximations differ significantly from the true value (roughly
0.693). Even moderate accuracy for $\ln{2}$ requires many terms of the Taylor
series, far beyond what pictures explain (Problem 4.20). The problem is
that x in $\ln{(1+x)}$ is 1, so the $x^{n}$ factor in each term of the Taylor series
does not shrink the high-n terms.}\\
\\
\Large \textrm{The same problem happens when computing $\pi$ using Leibniz’s arctangent
series (Section 4.2.3)}\\
\begin {equation}\label{formula4.36}
\arctan x=x-\frac{x^{3}}{3}+\frac{x^{5}}{5}-\frac{x^{7}}{7}\cdots\cdot
\end{equation}
\Large \textrm{By using $x = 1 $, the direct approximation of $\pi$/4 requires many terms
to attain even moderate accuracy. Fortunately, the trigonometric identity
$\arctan 1=4\arctan1/5-\arctan1/239$ lowers the largest x to 1/5 and
thereby speeds the convergence.}\\
\\
\large \textsl{Is there an analogous that helps estimate $\ln{2}$ ?}\\
\\
\Large \textrm{Because 2 is also (4/3)/(2/3), an analogous rewriting of $\ln{2}$ is}\\
\begin {equation}\label{formula4.37}
\ln{2}=\ln{\frac{4}{3}}-\ln{\frac{2}{3}}\cdot
\end{equation}
\end{document}
